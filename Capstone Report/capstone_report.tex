\documentclass{article}
\usepackage{hyperref}
\usepackage{enumerate}
\usepackage[style=authoryear]{biblatex}
\addbibresource{citations.bib}

\title{Machine Learning Engineer Nanodegree \\
\large Capstone Project}
\author{Haitham Alhad Hyder}

\begin{document}

\maketitle

\section{Definition}\label{definition}

Note:\emph{(approx. 1-2 pages)}

\subsection*{Project Overview}\label{project-overview}

The project involves the infamous Titanic sinkage of 1912. The data set
and inspiration to solve the problem are from the Titanic challenge
Kaggle competition \parencite{kaggle}.

Only 1502 of the 2224 on board the ship survived and was a big tragedy
since it was labelled the ``unsinkable'' ship. We will be building a
model that will identify the chance of survival a person has.

\subsection{Problem Statement}\label{problem-statement}

We will be building a model that inputs the characteristics of a person
and outputs a value between 0 and 1 giving us the chance of someone
surviving.

The model that we come up with should be capable of identifying patterns
that affect the chance of survival and therefore, by rounding the
probability it outputs we can tell if that person's prediction is
survival or not.

The steps to solving this problem are as follows:

\begin{enumerate}
\item
  Loading up the train data and split it into a train, validation and
  test data sets.
\item
  Build an XGBoost binary linear classifier as well as a PyTorch neural
  network.
\item
  At the same time we build an SKLearn decision tree which will be our
  Statistical base model that will help us compare it to our ML(Machine
  Learning) models.
\item
  After evaluating all the models and picking the best one, we will
  deploy to an API endpoint.
\item
  Finally we will create a web page that communicates with our model;
  sending in a user's chosen passenger characteristics and outputting
  the probability of survival.
\end{enumerate}

\subsection{Metrics}\label{metrics}

In this section, you will need to clearly define the metrics or
calculations you will use to measure performance of a model or result in
your project. These calculations and metrics should be justified based
on the characteristics of the problem and problem domain. Questions to
ask yourself when writing this section:

\begin{itemize}
\item
  \emph{Are the metrics you've chosen to measure the performance of your
  models clearly discussed and defined?}
\item
  \emph{Have you provided reasonable justification for the metrics
  chosen based on the problem and solution?}
\end{itemize}

\section{Analysis}

Note: \emph{(approx. 2-4 pages)}

\subsection{Data Exploration}\label{data-exploration}

In this section, you will be expected to analyze the data you are using
for the problem. This data can either be in the form of a data set (or
data sets), input data (or input files), or even an environment. The
type of data should be thoroughly described and, if possible, have basic
statistics and information presented (such as discussion of input
features or defining characteristics about the input or environment).
Any abnormalities or interesting qualities about the data that may need
to be addressed have been identified (such as features that need to be
transformed or the possibility of outliers). Questions to ask yourself
when writing this section:

\begin{itemize}
\item
  \emph{If a data set is present for this problem, have you thoroughly
  discussed certain features about the data set? Has a data sample been
  provided to the reader?}
\item
  \emph{If a data set is present for this problem, are statistics about
  the data set calculated and reported? Have any relevant results from
  this calculation been discussed?}
\item
  \emph{If a data set is \textbf{not} present for this problem, has
  discussion been made about the input space or input data for your
  problem?}
\item
  \emph{Are there any abnormalities or characteristics about the input
  space or data set that need to be addressed? (categorical variables,
  missing values, outliers, etc.)}
\end{itemize}

\subsection{Exploratory
Visualization}\label{exploratory-visualization}

In this section, you will need to provide some form of visualization
that summarizes or extracts a relevant characteristic or feature about
the data. The visualization should adequately support the data being
used. Discuss why this visualization was chosen and how it is relevant.
Questions to ask yourself when writing this section:

\begin{itemize}
\item
  \emph{Have you visualized a relevant characteristic or feature about
  the data set or input data?}
\item
  \emph{Is the visualization thoroughly analyzed and discussed?}
\item
  \emph{If a plot is provided, are the axes, title, and datum clearly
  defined?}
\end{itemize}

\subsection{Algorithms and
Techniques}\label{algorithms-and-techniques}

In this section, you will need to discuss the algorithms and techniques
you intend to use for solving the problem. You should justify the use of
each one based on the characteristics of the problem and the problem
domain. Questions to ask yourself when writing this section:

\begin{itemize}
\item
  \emph{Are the algorithms you will use, including any default
  variables/parameters in the project clearly defined?}
\item
  \emph{Are the techniques to be used thoroughly discussed and
  justified?}
\item
  \emph{Is it made clear how the input data or data sets will be handled
  by the algorithms and techniques chosen?}
\end{itemize}

\subsection{Benchmark}\label{benchmark}

In this section, you will need to provide a clearly defined benchmark
result or threshold for comparing across performances obtained by your
solution. The reasoning behind the benchmark (in the case where it is
not an established result) should be discussed. Questions to ask
yourself when writing this section:

\begin{itemize}
\item
  \emph{Has some result or value been provided that acts as a benchmark
  for measuring performance?}
\item
  \emph{Is it clear how this result or value was obtained (whether by
  data or by hypothesis)?}
\end{itemize}

\section{Methodology}\label{methodology}

Note: \emph{(approx. 3-5 pages)}

\subsection{Data Preprocessing}\label{data-preprocessing}

In this section, all of your preprocessing steps will need to be clearly
documented, if any were necessary. From the previous section, any of the
abnormalities or characteristics that you identified about the data set
will be addressed and corrected here. Questions to ask yourself when
writing this section:

\begin{itemize}
\item
  \emph{If the algorithms chosen require preprocessing steps like
  feature selection or feature transformations, have they been properly
  documented?}
\item
  \emph{Based on the \textbf{Data Exploration} section, if there were
  abnormalities or characteristics that needed to be addressed, have
  they been properly corrected?}
\item
  \emph{If no preprocessing is needed, has it been made clear why?}
\end{itemize}

\subsection{Implementation}\label{implementation}

In this section, the process for which metrics, algorithms, and
techniques that you implemented for the given data will need to be
clearly documented. It should be abundantly clear how the implementation
was carried out, and discussion should be made regarding any
complications that occurred during this process. Questions to ask
yourself when writing this section:

\begin{itemize}
\item
  \emph{Is it made clear how the algorithms and techniques were
  implemented with the given data sets or input data?}
\item
  \emph{Were there any complications with the original metrics or
  techniques that required changing prior to acquiring a solution?}
\item
  \emph{Was there any part of the coding process (e.g., writing
  complicated functions) that should be documented?}
\end{itemize}

\subsection{Refinement}\label{refinement}

In this section, you will need to discuss the process of improvement you
made upon the algorithms and techniques you used in your implementation.
For example, adjusting parameters for certain models to acquire improved
solutions would fall under the refinement category. Your initial and
final solutions should be reported, as well as any significant
intermediate results as necessary. Questions to ask yourself when
writing this section:

\begin{itemize}
\item
  \emph{Has an initial solution been found and clearly reported?}
\item
  \emph{Is the process of improvement clearly documented, such as what
  techniques were used?}
\item
  \emph{Are intermediate and final solutions clearly reported as the
  process is improved?}
\end{itemize}

\hypertarget{iv.-results}{%
\subsection{IV. Results}\label{iv.-results}}

Note: \emph{(approx. 2-3 pages)}

\subsection{Model Evaluation and
Validation}\label{model-evaluation-and-validation}

In this section, the final model and any supporting qualities should be
evaluated in detail. It should be clear how the final model was derived
and why this model was chosen. In addition, some type of analysis should
be used to validate the robustness of this model and its solution, such
as manipulating the input data or environment to see how the model's
solution is affected (this is called sensitivity analysis). Questions to
ask yourself when writing this section:

\begin{itemize}
\item
  \emph{Is the final model reasonable and aligning with solution
  expectations? Are the final parameters of the model appropriate?}
\item
  \emph{Has the final model been tested with various inputs to evaluate
  whether the model generalizes well to unseen data?}
\item
  \emph{Is the model robust enough for the problem? Do small
  perturbations (changes) in training data or the input space greatly
  affect the results?}
\item
  \emph{Can results found from the model be trusted?}
\end{itemize}

\subsection{Justification}\label{justification}

In this section, your model's final solution and its results should be
compared to the benchmark you established earlier in the project using
some type of statistical analysis. You should also justify whether these
results and the solution are significant enough to have solved the
problem posed in the project. Questions to ask yourself when writing
this section:

\begin{itemize}
\item
  \emph{Are the final results found stronger than the benchmark result
  reported earlier?}
\item
  \emph{Have you thoroughly analyzed and discussed the final solution?}
\item
  \emph{Is the final solution significant enough to have solved the
  problem?}
\end{itemize}

\section{Conclusion}\label{conclusion}

Note: \emph{(approx. 1-2 pages)}


\subsection{Free-Form Visualization}\label{free-form-visualization}

In this section, you will need to provide some form of visualization
that emphasizes an important quality about the project. It is much more
free-form, but should reasonably support a significant result or
characteristic about the problem that you want to discuss. Questions to
ask yourself when writing this section:

\begin{itemize}
\item
  \emph{Have you visualized a relevant or important quality about the
  problem, data set, input data, or results?}
\item
  \emph{Is the visualization thoroughly analyzed and discussed?}
\item
  \emph{If a plot is provided, are the axes, title, and datum clearly
  defined?}
\end{itemize}

\subsection{Reflection}\label{reflection}

In this section, you will summarize the entire end-to-end problem
solution and discuss one or two particular aspects of the project you
found interesting or difficult. You are expected to reflect on the
project as a whole to show that you have a firm understanding of the
entire process employed in your work. Questions to ask yourself when
writing this section:

\begin{itemize}
\item
  \emph{Have you thoroughly summarized the entire process you used for
  this project?}
\item
  \emph{Were there any interesting aspects of the project?}
\item
  \emph{Were there any difficult aspects of the project?}
\item
  \emph{Does the final model and solution fit your expectations for the
  problem, and should it be used in a general setting to solve these
  types of problems?}
\end{itemize}

\subsection{Improvement}\label{improvement}

In this section, you will need to provide discussion as to how one
aspect of the implementation you designed could be improved. As an
example, consider ways your implementation can be made more general, and
what would need to be modified. You do not need to make this
improvement, but the potential solutions resulting from these changes
are considered and compared/contrasted to your current solution.
Questions to ask yourself when writing this section:

\begin{itemize}
\item
  \emph{Are there further improvements that could be made on the
  algorithms or techniques you used in this project?}
\item
  \emph{Were there algorithms or techniques you researched that you did
  not know how to implement, but would consider using if you knew how?}
\item
  \emph{If you used your final solution as the new benchmark, do you
  think an even better solution exists?}
\end{itemize}

\begin{center}\rule{0.5\linewidth}{\linethickness}\end{center}

\textbf{Before submitting, ask yourself. . .}

\begin{itemize}
\item
  Does the project report you've written follow a well-organized
  structure similar to that of the project template?
\item
  Is each section (particularly \textbf{Analysis} and
  \textbf{Methodology}) written in a clear, concise and specific
  fashion? Are there any ambiguous terms or phrases that need
  clarification?
\item
  Would the intended audience of your project be able to understand your
  analysis, methods, and results?
\item
  Have you properly proof-read your project report to assure there are
  minimal grammatical and spelling mistakes?
\item
  Are all the resources used for this project correctly cited and
  referenced?
\item
  Is the code that implements your solution easily readable and properly
  commented?
\item
  Does the code execute without error and produce results similar to
  those reported?
\end{itemize}

\newpage
\printbibliography

\end{document}